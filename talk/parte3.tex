\documentclass[]{beamer}
\usetheme{Boadilla}
\usecolortheme{beaver}
\usepackage[utf8]{inputenc}
\usepackage[american]{babel}
\usepackage[T1]{fontenc}
\usepackage{color}
\usepackage{amsmath}

\newcommand{\semitransp}[2][35]{{\color{fg!#1}#2}}
\newcommand{\ttt}[1]{\texttt{#1}}
\newcommand{\Na}{\mathbb{N}}
\newcommand{\R}{\mathbb{R}}

% Requests the 'color' package
\newcommand{\code}[1]{\colorbox[gray]{0.8}{\texttt{#1}}}

\definecolor{darkgreen}{rgb}{0,0.5,0}
\definecolor{navy blue}{RGB}{0, 0, 99}
\definecolor{rock blue}{RGB}{156, 170, 198}
\definecolor{refs}{gray}{0.7}

\setbeamercolor{attention box}{use={title in head/foot},%
  fg=navy blue, %
  bg=title in head/foot.bg
}

\author[Sobral]{Francisco N. C. Sobral e Júlia Guizardi\\ Universidade Estadual de Maringá}

\date[XXIX SEMAT]{XXXII Semana da Matemática\\UEM, 2022}

\title[Mat. Mult.]{Fatoração LU e resolução de sistemas}

% Remove shadows in blocks
\setbeamertemplate{blocks}[rounded][shadow=false]

% Coisas invisiveis ficam meio transparentes
\setbeamercovered{transparent=40} 

% Desabilita os menuzinhos chatos
\setbeamertemplate{navigation symbols}{} 

% Faz com que as tabelas e figuras sejam numeradas
%\setbeamertemplate{caption}[numbered]

% Lists
% Set circles, triangles and change colors
\setbeamertemplate{itemize items}[triangle]
\setbeamertemplate{enumerate items}[circle]
\setbeamertemplate{section in toc}[circle]
\setbeamertemplate{subsection in toc}
{
  \leavevmode
  \leftskip=2em
  {\usebeamercolor[bg]{item projected}$\bullet$}
  \hskip0.5em\inserttocsubsection\par
}

\setbeamercolor{item projected}{ %
  fg=rock blue, %
  bg=navy blue %
}

\setbeamercolor{item}{ %
  fg=navy blue %
}



\begin{document}

\begin{frame}[plain]
\titlepage
\end{frame}

\begin{frame}{Fatoração LU e resolução de sistemas}
	Dado um sistema matricial da forma 
	\begin{equation}\label{sistema}
	Ax=b,
	\end{equation}
	podemos calcular uma matriz triangular inferior unitária $L$ e outra triangular superior $U,$ tal que $A=LU.$ Assim, para resolver (\ref{sistema}), operamos
	$$Ly=b \text{  e  } Ux=y,$$
	$$\Rightarrow Ax=LUx=Ly=b.$$
\end{frame}

\begin{frame}{Fatoração LU}
	\begin{equation}
	A=LU
	\end{equation}
	
	$$
	\begin{bmatrix}
	a_{1,1} & \cdots &  & a_{1,n}\\
	\vdots & \ddots  & & \vdots\\
	& &  & \\
	a_{n,1} & \cdots &  & a_{n,n}
	\end{bmatrix}
	=
	\begin{bmatrix}
	1 & 0 & \cdots & 0\\
	\ell_{2,1} & 1 & & \vdots\\
	\vdots & & \ddots & 0\\
	\ell_{n,1} & \cdots & \ell_{n,n-1} & 1
	\end{bmatrix}\begin{bmatrix}
	u_{1,1} & u_{1,2} & \cdots & u_{1,n}\\
	0 & u_{2,2} & & \vdots\\
	\vdots & & \ddots & \\
	0 & \cdots & 0 & u_{n,n}
	\end{bmatrix}
	$$
\end{frame}

\begin{frame}{Exemplo 1}
Seja a matriz $A,$
$$\begin{bmatrix}6&18&22\\ 4&7&7\\ 2&2&2\end{bmatrix}$$
Primeiramente, vamos calcular a fatoração $LU.$
$$\ell_{2,2} = \dfrac{a_{2,1}}{a_{1,1}}= \dfrac{4}{6}$$
Vamos operar $linha_2 = linha_2 - \ell_{2,2} \cdot linha_1,$
$$\Rightarrow \begin{bmatrix}6&18&22\\ 0&-5&-\frac{23}{3}\\ 2&2&2\end{bmatrix}.$$
\end{frame}

\begin{frame}{Exemplo 1}
Vamos repetir o processo até zeramos toda a parte inferior da matriz.
$$\ell_{3,1} = \dfrac{a_{3,1}}{a_{1,1}} = \dfrac{2}{6}$$
Daí $linha_3 = linha_3 - \ell_{3,1} \cdot linha_1: $
$$\Rightarrow \begin{bmatrix}6&18&22\\ 0&-5&-\frac{23}{3}\\ 0&-4&-\frac{16}{3}\end{bmatrix}.$$
\end{frame}

\begin{frame}{Exemplo 1}
Por fim, $$\ell_{3,2}  = \dfrac{-4}{-5}$$
Daí $linha_3 = linha_3 - \ell_{3,2} \cdot linha_2: $
$$\Rightarrow \begin{bmatrix}6&18&22\\ 0&-5&-\frac{23}{3}\\ 0&0&\frac{4}{5}\end{bmatrix}.$$
\end{frame}

\begin{frame}{Exemplo 1}

Com isso temos que $$L=\begin{bmatrix}1&0&0\\ \frac{4}{6}&1&0\\ \frac{2}{6}&\frac{4}{5}&1\end{bmatrix} U = \begin{bmatrix}6&18&22\\ 0&-5&-\frac{23}{3}\\ 0&0&\frac{4}{5}\end{bmatrix}.$$ 
\end{frame}

\begin{frame}{Porque fazer?}
Vantagens da fatoração:
\begin{itemize}
	\item Apesar de serem dois sistemas, tais sistemas são simples de serem resolvidos pela quantidade de entradas nulas, uma vez que as matrizes $L$ e $U$ são triangulares;
	
	\item  Caso seja necessário resolver outros sistemas lineares com a mesma matriz de coeficientes $A,$ mas diferentes valores para $b,$ podemos reaproveitar a fatoração de $A;$
	
	\item A fatoração $LU$ nos permite executar a mudança de uma coluna de $A$ de forma eficiente e sem ter que repetir toda a fatoração;
	
	\item Podemos fazer a fatoração usando pouco espaço da máquina.
\end{itemize}

\end{frame}

\begin{frame}{Armazenamento}
Note que é possível armazenar $L$ e $U$ em uma única matriz, consideremos $n =3:$
$$L=\begin{bmatrix}1&0&0\\ \ell_{2,1}&1&0\\ \ell_{3,1}&\ell_{3,2}&1\end{bmatrix} U = \begin{bmatrix}u_{1,2}&u_{1,2}&u_{1,3}\\ 0&u_{2,2}&u_{2,3}\\ 0&0&u_{3,3}\end{bmatrix}$$ 
O que podemos fazer é criar uma única matriz da forma:
$$\begin{bmatrix}u_{1,2}&u_{1,2}&u_{1,3}\\ \ell_{2,1}&u_{2,2}&u_{2,3}\\ \ell_{3,1}&\ell_{3,2}&u_{3,3}\end{bmatrix}$$ 
Isso é possível já que $L$ é uma matriz unitária.
\end{frame}



\begin{frame}{Resolução de sistema}
	Com a fatoração em mãos, basta encontramos $y$ em
	$$Ly=b. $$
	E depois, encontramos $x,$
	$$ Ux=y.$$
\end{frame}

\begin{frame}{Exemplo 2}
Seja o sistema matricial dado por $Ax=b,$
$$\begin{bmatrix}6&18&22\\ 4&7&7\\ 2&2&2\end{bmatrix}\cdot \begin{bmatrix}x_1\\ x_2\\ x_3\end{bmatrix}=\begin{bmatrix}12\\ 24\\ 12\end{bmatrix}.$$

Já calculamos a fatoração $LU$ da matriz $A$ no Exemplo 1. Agora, vamos resolver $Ly=b:$

$$\begin{bmatrix}1&0&0\\ \frac{4}{6}&1&0\\ \frac{2}{6}&\frac{4}{5}&1\end{bmatrix}\cdot \begin{bmatrix}y_1\\ y_2\\ y_3\end{bmatrix}=\begin{bmatrix}12\\ 24\\ 12\end{bmatrix}$$

\end{frame}

\begin{frame}{Exemplo 2}
É fácil ver que $y_1=12,$ os outros valores calculamos substituindo. Observe que são contas simples, e obtemos $$y=\begin{bmatrix}12\\ 16\\ -\frac{24}{5}\end{bmatrix}.$$

Por fim, vamos resolver $Ux=y.$
$$\begin{bmatrix}6&18&22\\ 0&-5&-\frac{23}{3}\\ 0&0&\frac{4}{5}\end{bmatrix}\cdot \begin{bmatrix}x_1\\ x_2\\ x_3\end{bmatrix}=\begin{bmatrix}12\\ 16\\ -\frac{24}{5}\end{bmatrix}$$
\end{frame}


\begin{frame}{Exemplo 2}
Fazendo a substituição encontramos  $$x=\begin{bmatrix}6\\ 6\\ -6\end{bmatrix}.$$
\end{frame}

\begin{frame}{Pivoteamento}

\end{frame}

\begin{frame}[plain, noframenumbering]
\vfill
\vfill
\vfill
\begin{center}
	\Huge Obrigada!
\end{center}
\vfill
\vfill

\end{frame}
\end{document}